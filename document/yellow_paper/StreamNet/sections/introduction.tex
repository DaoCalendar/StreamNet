\section{Introduction}
Ever since bitcoin \cite{nakamoto2008bitcoin} has been proposed, blockchain technology has been widely studied for $10$ years. 
Extensive adoptions of blockchain technologies was seen in real world applications such as 
financial services with potential regulation challenges \cite{michael2018blockchain, tapscott2017blockchain}, 
supply chains \cite{korpela2017digital,tian2016agri, abeyratne2016blockchain}, 
health cares \cite{azaria2016medrec,yue2016healthcare} and IOT devices \cite{christidis2016blockchains}.
The core of blockchain technology depends on the consensus algorithms applying to the open distributed computing world.
Where computers can join and leave the network and these computers can cheat.

As the first protocol that can solve the so called Byzantine general's problem, 
bitcoin system suffers from the problem of low transaction rate with a transaction per second (TPS) of approximately $7$, and long confirmation time (about an hour).
As more and more machines joined the network, they are competing for the privileges to attach the block (miners) which result in huge waste of electric power.
While sky rocketing fees are payed to make sure the transfers of money will be placed in the chain.
On par, there are multiple proposals to solve the low transaction speed issue. 

One method intends to solve the speed problem without changing the chain data structure, for instance, 
the bitcoin cash (BCH) fork of bitcoin (BTC) system tries to improve the throughput of the system by enlarging the data size of each block from $1$ Mb to $4$ Mb. 
To minimize the cost of POW, a proof of stake method POS \cite{wood2014ethereum} is proposed to make sure that only those who in the network have the privilege to attach the block only if they have a large amount of token shares.
Another idea targeting at utilizing the power in POW to do useful and meaningful tasks such as training machine learning models are also proposed \cite{matthew2017aion}.
In addition, inspired by the PBFT algorithm \cite{castro1999practical} and a set of its related variations, so called hybrid (or consortium) chain was proposed. 
The general idea is to use two step algorithm, the first step is to elect a committee, the second step is collecting committee power to employ PBFT for consensus.
Bitcoin-NG \cite{eyal2016bitcoin} is the early adopter of this idea, which splits the blocks of bitcoin into two groups, one is for master election and another for regular transaction blocks. 
Honey-badger \cite{miller2016honey} is the system that firstly introduced the consensus committee, it uses a predefined members to perform PBFT algorithm to reach consensus.  
The Byzcoin system \cite{kogias2016enhancing} brought forth the idea of POW for the committee election, and uses a variation of PBFT called collective signing for speed purposes.
The Algorand \cite{gilad2017algorand} utilizes a random function to anonymously elect committee and use this committee to commit blocks, and the member of the committee only have one chance to commit block.
All these systems have one common feature, the split of layers of players in the network, which results in the complexity of the implementation of the system.

While aforementioned methods are trying to avoid side chains, another thread of effort is put on using direct acyclic graph DAG to merge side chains.
The first ever idea comes with growing the blockchain with trees instead of chains \cite{sompolinsky2013accelerating}, which results in the well known GHOST protocol \cite{sompolinsky2015secure}.
If one block links to $\geq 2$ previous blocks, then the data structure grows like a DAG instead of tree \cite{sompolinsky2016spectre, sompolinskyphantom, lewenberg2015inclusive}.
There is a improvement of the GHOST based DAG algorithm which can achieve $6000$ of TPS in reality \cite{li2018scaling}.
Another set of methods tried to avoid finality of constructing a linear total order by introducing the probability of confirmation in the network \cite{popov2016tangle, churyumov2016byteball}. 
However, suffering from engineering issues, mainly due to the lack of transaction frequency and the growing complexity due to the network expansion.
These system in reality are rely on centralized methods to maintain their stability.
Some of the side chain methods also borrows the idea of DAG, such as nano \cite{lemahieu2018nano} and vite \cite{liuvite}.

Social network analysis has widely adopted the method of streaming graph computing \cite{ediger2011tracking, green2012fast, ediger2012stinger} which deals
with how to quickly maintain information on a temporally or spatially changing graph without traversing the whole graph. 
We view the DAG based method as a streaming graph problem which is about how to compute the total order and achieve consensus without consuming more computing power.
In distributed database systems, the problem of replicating data across machines is a well studied topic \cite{demers1988epidemic}.
Due to the low efficiency of the bitcoin network, there are multiple ways to accelerate the message passing efficiency \cite{klarmanbloxroute}, however, they did not deal with the network complexity. 
We view the problem of scaling DAG system in the network of growing size and topological complexity as another challenging issue, and proposed our own gossip solution.
The main contribution of this paper is how to utilize the streaming graph analysis methods and new gossip protocol to bring the DAG systems into real decentralized, and stabilized growing system.
