To achieve high throughtput in the POW based blockchain systems, a series of methods has been proposed, 
and DAG is one of the most active and promising field.
%Considering major issues in the existing DAG systems such as centralization,
%double-spending, sybil attacks, and slow transaction validation speed etc. 
We designed and implemented the StreamNet aiming to engineering 
a scalable and endurable DAG system. 
When attaching a new block in the DAG, only two tips are selected. 
One is the `parent' tip whose definition is the same as in Conflux \cite{li2018scaling},
another is using Markov Chain Monte Carlo (MCMC) technique by which the definition is the same as IOTA \cite{popov2016tangle}.
By leveraging the graph streaming property,
we infer a pivotal chain along the path of each epoch in the graph,
and a total order of the graph could be calculated without a centralized authority and high transaction validation speed will be achieved even if the DAG is growing.
%and each block in the pivotal chain will have the highest Katz centrality score (as oppsed to the $GHOST$ rule).
%with Katz centrality as the weight determinator. 
%In addtion, with comprehensive integration of information such as transaction (vertices), 
%transaction approval information (edges), and network structure (such as community structure) 
%a tip filtering algorithm can be optionally configured to detect and avoid double spend / sybil attacks.
